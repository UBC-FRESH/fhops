% AUTO-GENERATED from motivation_story.md -- do not edit directly.
Forest harvest-planning software still leans on bespoke, closed
toolchains that make it hard for regulators, Indigenous governments, and
researchers to audit models or extend them for emerging policy
questions. Jaffray et\,al.~(2025, submitted to the \emph{International
Journal of Forest Engineering}) catalogue the recurring pain points:
one-off solver integrations, weak telemetry, limited robustness testing,
and siloed datasets that rarely ship with reproducible scripts. FHOPS
exists to close those gaps for B.C. operations and comparable
jurisdictions.

The SoftwareX paper will highlight three gaps we actively address:

\begin{enumerate}
\def\labelenumi{\arabic{enumi}.}
\tightlist
\item
  \textbf{Open, reusable tooling.} FHOPS publishes its data contract,
  CLI, and solver implementations under MIT so other teams can ingest
  the same scenarios, swap heuristics, and contribute modules without
  vendor lock-in. The scenario schema mirrors what forestry engineers
  already use in practice (blocks, machines, landings, shifts), but the
  implementation is scriptable and version-controlled.
\item
  \textbf{Integrated workflow + automation.} Instead of the ad hoc
  ``optimizer + spreadsheet'' pattern flagged in the review, FHOPS
  provides deterministic solvers (Pyomo+HiGHS), SA/ILS/Tabu heuristics,
  a turnkey tuning harness, and telemetry/playback tooling that runs
  from \texttt{make\ assets}. Every figure/table in the manuscript will
  be regenerated from the same scripts users run locally.
\item
  \textbf{Robust evaluation + extensibility.} FHOPS layers stochastic
  playback, stress testing, and cost models over the base scheduler so
  we can quantify solution stability before fielding new policies. Those
  evaluation hooks also set up Rosalia's thesis Chapter 2: she retains
  the full BC case studies (two--three tenures) and uses FHOPS as the
  engine, while this SoftwareX paper focuses on the platform itself.
\end{enumerate}

\begin{quote}
Reuse plan: exporter script will render this Markdown into
\texttt{sections/includes/motivation\_story.tex} for the manuscript and
\texttt{docs/overview\_shared\_motivation.rst} for Sphinx so the same
paragraphs stay synchronized.
\end{quote}
