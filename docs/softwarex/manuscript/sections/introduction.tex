% Section 1: Motivation and significance
\section{Motivation and significance}

Modern forest harvesting optimisation problems mirror the increasing complexity highlighted in prior SoftwareX exemplars: planners need transparent tooling that spans planning horizons, integrates heterogeneous data sources, and supports reproducible experimentation \cite{Lyden2021PyLESA,Schwarz2021pycity}. FHOPS targets this gap by packaging the optimisation heuristics, MIP constructs, and evaluation telemetry we already operate in production into a single, research-friendly stack. Like PyLESA, we frame the work around concrete gaps (e.g., multi-problem workflows, reproducible telemetry, solver governance) and demonstrate how FHOPS compresses the “scoping → benchmarking → deployment” loop for harvesting teams.

In the rest of this manuscript we follow the SoftwareX template: Section~\ref{sec:software-description} decomposes the architecture, Section~\ref{sec:illustrative-example} walks through a benchmark suite spanning three optimisation classes, and Section~\ref{sec:impact} quantifies community adoption plus governance processes inspired by libxc/MOOSE. We close with the roadmap and submission logistics to keep FHOPS aligned with the reproducibility signals SoftwareX emphasises \cite{Blauth2023cashocs}.
