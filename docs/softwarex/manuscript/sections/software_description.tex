% Section 2: Software description
\section{Software description}
\label{sec:software-description}

\textbf{Overview.} FHOPS ships as a Python package backed by Typer-based CLI tooling and a data contract that keeps datasets, solver inputs, and telemetry consistent. The system breaks down into three layers:

\begin{enumerate}
  \item \textbf{Data ingest \& scenario contract} – schema + loaders that convert CSV/YAML datasets into typed \texttt{Scenario} objects.
  \item \textbf{Optimisation services} – heuristics (SA/ILS/Tabu), MIP builder, and evaluation harness.
  \item \textbf{Telemetry \& assets} – reproducible logging, benchmark scripts, figure/table generation.
\end{enumerate}

Each layer is consumable via API or CLI, mirroring the dual entry points emphasized in pycity\_scheduling, TSFEL, and contemporary forest-planning DSS (e.g., PRISM, multi-attribute spatial tools) \cite{nguyen2022prism,kuhmaier2010dss}.

Figure~\ref{fig:fhops-prisma-overview} summarises the automation pipeline that underpins this layered design. The PRISMA-style diagram mirrors the WS3 EI manuscript approach: inputs (scenario contract, curated datasets, automation configs) flow through the FHOPS core (ingest, solver/tuning, telemetry) before automation scripts export shared assets that power both the SoftwareX submission and the Sphinx docs. The include lives under \texttt{sections/includes/} so the same source file can be refreshed via \texttt{make assets} along with the other shared snippets.

\begin{figure}[t]
  \centering
  \prismaflowstart
  % Inputs
  \prismaflownode{in1}{below=of tc}{\textbf{Scenario contract}\\Blocks, machines, landings, calendar CSV + scenario YAML}{}
  \prismaflownode{in2}{right=of in1}{\textbf{Reference \& synthetic datasets}\\`examples/*`, `run_dataset_inspection.py`}{}
  \prismaflownode{in3}{right=of in2}{\textbf{Automation configs}\\Make targets, CI hooks, contribution rules}{}
  \prismalabel{1.3*\mh}{in1.west}{Inputs}

  % FHOPS core
  \prismaflownode{core1}{below=of in2}{\textbf{Data ingestion}\\Validation, schema checks, synthetic tier generation}{in2}
  \prismaflownode{core2}{below=of core1}{\textbf{Solver core}\\Pyomo MIP + SA/ILS/Tabu heuristics, tuning harness}{core1}
  \prismaflownode{core3}{right=of core2}{\textbf{Evaluation + telemetry}\\Playback, KPI reporting, costing, scaling}{core2}
  \prismalabel{1.3*\mh}{in1.west |- {$(core1)!0.5!(core2)$}}{FHOPS core}

  % Asset exports
  \prismaflownode{int1}{below=of core2}{\textbf{Automation assets}\\Benchmarks, tuning tables, playback summaries, cost + scaling data}{core2}
  \prismaflownode{int2}{right=of int1}{\textbf{Shared snippets}\\Markdown/CSV includes $\rightarrow$ LaTeX+RST (`export\_docs\_assets.py`)}{core3}
  \prismalabel{1.3*\mh}{in1.west |- int1}{Shared artifacts}

  % Outputs
  \prismaflownode{out1}{below=of int1}{\textbf{Outputs}\\SoftwareX manuscript, reproducible figures, Sphinx docs, submission bundle}{int1}
  \prismalabel{1.3*\mh}{in1.west |- out1}{Outputs}

  % Arrows
  \prismaflowarrow{in1}{core1}
  \prismaflowarrow{in2}{core1}
  \prismaflowarrow{in3}{core1}
  \prismaflowarrow{core1}{core2}
  \prismaflowarrow{core2}{core3}
  \prismaflowarrow{core2}{int1}
  \prismaflowarrow{core3}{int2}
  \prismaflowarrow{int1}{out1}
  \prismaflowarrow{int2}{out1}
  \prismaflowend
  \caption{FHOPS SoftwareX automation pipeline. Inputs (scenario contract + curated datasets) feed the FHOPS core (validation, solver stack, evaluation tooling). Asset scripts emit shared tables/figures consumed by the manuscript, submission package, and Sphinx docs.}
  \label{fig:fhops-prisma-overview}
\end{figure}


\subsection{Data ingest and scenario contract}
\textbf{Components:}
\begin{itemize}
  \item \texttt{fhops.scenario.contract} – Pydantic models for blocks, corridors, harvest systems, road jobs (aligned with BC operational constraints described by \citet{lahrsen2022productivity} and \citet{becker2018lidar}).
  \item CSV/Parquet loaders (\texttt{fhops.cli.dataset}) – generate scenarios from published datasets (TN-147 skyline corridors, TR-122 Roberts Creek blocks, community-forest bundles).
  \item Synthetic dataset generator (\texttt{fhops.scenario.synthetic}) – reproducible randomised scenarios with stratified harvest-system mixes.
  \item Future hooks for Rosalia Jaffray’s Chapter 2 datasets (community forests, skyline corridors, salvage plans); these remain private until the thesis publishes but are wired into the loader interfaces.
\end{itemize}

\textbf{Dependencies:} Pydantic, pandas, numpy.

\textbf{Outputs:} Validated \texttt{Scenario} objects with metadata (harvest system IDs, road-construction tables) plus telemetry-ready manifests.

\subsection{Optimisation services}
\textbf{Components:}
\begin{itemize}
  \item Heuristics under \texttt{fhops.optimization.heuristics} (SA, ILS, Tabu) with Typer CLI entry points (\texttt{fhops.cli.benchmarks}). Productivity lookup tables embed empirical studies \cite{lahrsen2022productivity,becker2018lidar} so planners can attach LiDAR-derived corridor metrics or dealer-grade payload charts without reinventing coefficients.
  \item MIP builder (\texttt{fhops.optimization.mip}) for shift-based scheduling when exact solutions are required (informed by formulations surveyed in \cite{paradis2018bilevel,nelson2003forestmodels}).
  \item Evaluation playback + KPI generation (\texttt{fhops.evaluation.playback}, \texttt{fhops.evaluation.metrics}).
\end{itemize}

\textbf{Dependencies:} numpy, scipy, OR-Tools/PuLP (MIP), Typer/Rich for CLI ergonomics.

\textbf{Outputs:} Solver traces (JSON telemetry), CSV KPI summaries, and intermediate artifacts supporting reproducibility.

\subsection{Telemetry, assets, and automation}
\textbf{Components:}
\begin{itemize}
  \item Telemetry writer (\texttt{fhops.telemetry}) + CLI flags (e.g., \texttt{--telemetry-log}) recording inputs/outputs.
  \item Asset pipelines (planned) under \texttt{docs/softwarex/assets/} for figures/tables shared with Sphinx and the manuscript.
  \item Benchmark scripts (\texttt{fhops.cli.benchmarks run-suite}) orchestrating reproducible experiments for Section~\ref{sec:illustrative-example}.
\end{itemize}

\textbf{Automation:} Make targets in \texttt{docs/softwarex/manuscript/} run \texttt{latexmk}; future targets will execute benchmark scripts (\texttt{make manuscript-benchmarks}) prior to building so figures/tables remain in sync.
