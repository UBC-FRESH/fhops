% Section 4: Impact
\section{Impact}
\label{sec:impact}

\subsection{Addressing the gaps identified by Jaffray et\,al.}
The 2025 systematic review stressed that operational-planning studies rarely publish reusable code, almost never share telemetry, and seldom document how to extend their models beyond the initial case study. FHOPS answers those gaps directly: it formalises the scenario contract (blocks, machines, mobilisation, shifts), exposes the solver/evaluation stack via CLI and Python APIs, and version-controls every dataset used in the benchmarks. Because the schema mirrors what BC practitioners already use, the same bundle can serve research, regulatory, and community-forest planning needs without custom adapters.

\subsection{Reproducible benchmarking infrastructure}
Running \texttt{make manuscript-benchmarks} regenerates every dataset inspection, solver run, tuning study, playback analysis, costing demo, and synthetic scaling sweep. Each invocation appends a signed record (UTC timestamp, commit, SHA-256 hash) to \texttt{docs/softwarex/assets/benchmark\_runs.log}, and the resulting artefacts live under \texttt{docs/softwarex/assets/data/**}. Tables~\ref{tab:benchmarks}--\ref{tab:tuning} and Figures~\ref{fig:playback}--\ref{fig:scaling} are ingested directly from those CSV/JSON files, so other researchers—and future case-study authors—can cross-check every number against a concrete asset path. The same scripts power the public FHOPS documentation, keeping narrative, CLI help, and manuscript perfectly aligned.

\subsection{Project maturity and near-term scope}
FHOPS is a brand-new public release: \texttt{v1.0.0-beta1} is the first tag outside the UBC FRESH lab, and no external deployments have been run yet. That recency is intentional. The goal of publishing now is to lock in the reproducible tooling so that the upcoming BC case studies (and any future collaborators) inherit a stable CLI/API workflow rather than ad hoc notebooks. Internal pilots and solver validation exercises informed the architecture, but we wait to claim adoption metrics until partner studies are complete. In the meantime, the manuscript, benchmark log, and asset directories provide a single point of reference for anyone evaluating or extending the platform during its formative releases.

\subsection{Extensible deployments beyond British Columbia}
We are deliberately focusing the reference datasets on British Columbia deployments (ground-based CTL, skyline with tethered processors, helicopter and salvage programmes) because those partners are ready to validate FHOPS today. That does not mean the platform is limited to BC. The data contract, productivity/cost helper modules, and CLI plumbing were designed to accept new machine catalogs, block schemas, and currency conventions without touching the core solvers. The current release already imports regression models and cost tables from non-BC studies (e.g., FPInnovations, AFORA/ALPACA), demonstrating that adding external references is a configuration exercise rather than a refactor. As demand emerges elsewhere—whether that means other Canadian provinces or international tenure systems—we can extend FHOPS with the relevant units, currencies (e.g., on-the-fly conversion instead of the current CAD focus), and machine defaults instead of trying to anticipate every global context up front.

\subsection{Path to broader community uptake}
FHOPS ships under the MIT licence, accepts contributions via the public repository, and uses the same automation described above for everyday testing. That governance model makes it easy for regulators, Indigenous governments, or cooperatives to audit and extend the tooling without waiting for bespoke consulting engagements. When new datasets or helper modules land, contributors can add them through the shared CLI/API and immediately benefit from the reproducibility guarantees. In short, FHOPS provides the “common infrastructure” the literature has been missing: a maintained, auditable foundation that lowers the cost of publishing and comparing forest-harvest optimisation studies.
