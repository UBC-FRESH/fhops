% Section 5: Conclusions and future work
\section{Conclusions}
\label{sec:conclusions}

FHOPS translates a production-ready harvest-operations toolchain into a reproducible artifact. The manuscript demonstrates how the scenario contract, solver stack (Pyomo MIP plus SA/ILS/Tabu heuristics), and telemetry/playback services can be invoked via a single CLI/API workflow. Every figure and table (benchmarks, tuning leaderboards, playback robustness, scaling sweeps) now rebuilds from \texttt{make manuscript-benchmarks}, and the resulting artefacts—including the PRISMA workflow diagram—ship alongside the code so reviewers can audit both the narrative and the underlying CSV/JSON records.

Future work focuses on broadening validation and extending the helper library. The upcoming British Columbia deployments (ground-based CTL, skyline + tethered processors, helicopter/salvage programmes) will reuse the same automation pipeline to publish scenario bundles, solver runs, and KPI telemetry without duplicating tooling. On the modelling side we plan to flesh out forwarder/grapple-skidder productivity helpers (BC-adjusted AFORA/ALPACA regressions), add additional tuning profiles, and expand the synthetic generator so new harvest-system presets arrive with ready-made benchmark manifests (see \texttt{notes/softwarex\_manuscript\_change\_log.md} for the detailed backlog). Because these enhancements will follow the same reproducibility hooks described here, subsequent FHOPS releases can cite a single artifact while iterating on the domain-specific datasets.
